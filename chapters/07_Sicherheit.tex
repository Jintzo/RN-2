\section{Sicherheit}

\paragraph{Angriffe}
\begin{itemize}
	\item Klassischer Angreifer (Dolev-Yao): Omnipräsent im Netz
	\begin{itemize}
    \item kann Pakete abhören / manipulieren / eigene Pakete erzeugen
    \item kein Zugriff auf Endsysteme
    \item keine Entschlüsselung ohne Schlüssel
  \end{itemize}
	\item Abhören, Einfügen, Manipulieren, Man in the Middle, Replay, DoS, 
	\item[\( \leadsto \)] \textbf{System/Protokoll} verwendet \textbf{Bausteine} um \textbf{Schutzziele} zu realisieren und vor \textbf{Angriffen} zu schützen
\end{itemize}

\paragraph{Schutzziele (CIA)}
\begin{itemize}
  \item \emph{Anforderungen an eine Komponente oder ein System, die erfüllt werden müssen, um schützenswerte Güter vor Bedrohen zu schützen}
  \item \textbf{\underline{C}onfidentiality} (Vertraulichkeit): keine unautorisierte Informationsgewinnung (Bausteine: Asymmetrische/Symmetrische Verschlüsselung)
  \item \textbf{\underline{I}ntegrity} (Integrität): Kein Ersetzen, Einfügen oder Löschen von Daten
  \begin{itemize}
    \item \emph{starke Integrität}: Keine unautorisierte Manipulation von Daten möglich
    \item \emph{schwache Integrität}: Manipulation von Daten nicht \emph{unbemerkt} möglich
    \item Manipulationen in vielen Fällen nicht zu verhindern \( \leadsto \) schwache Integrität
    \item Bausteine: Tamper Proof-Module, Message Authentication Codes (MAC)
  \end{itemize}
   \item \textbf{\underline{A}vailability} (Verfügbarkeit): beschreibt, in welchem Maße die Systemfunktionalität von berechtigen Subjekten unabhängig von Einflüssen in Anspruch genommen werden kann
   \item \textbf{Authentizität}: angegebene Datenquelle ist tatsächl Quelle + Datenintegrität
   \begin{itemize}
    \item \emph{Subjektechtheit}: Bob will sicherstellen, dass er tatsächlich mit Alice spricht \\* \( \leadsto \) \emph{Authentifikation}
    \item \emph{Datenechtheit}: Bob will sicherstellen, dass Daten tatsächlich von Alice sind
    \item Bausteine: Zertifikate, Signaturen, gemeinsames Geheimnis
   \end{itemize}
  \item weitere Schutzziele: Privatheit, (Nicht-)Abstreitbarkeit
\end{itemize}

\paragraph{Verschlüsselung}
\begin{itemize}
	\item \textbf{symmetrisch}: Ent- und Verschlüsseln mit einem Schlüssel, sehr effizient
	\item \textbf{asymmetrisch}: Verschlüsseln: öffentlicher Schlüssel, Entschlüsseln: privater \\* \( \to \) RSA
\end{itemize}

\paragraph{Kryptografische Hashfunktion}
\begin{itemize}
	\item Einwegfunktion (\( H(m) \) effizient, \( H^{-1}(c) \) nicht)
	\item Zu gegebenen \( b \) schwierig, \( a \) zu finden mit \( H(a) = b \)
	\item Schwache Kollisionsresistenz: Zu \( a \) ein \( a' \) mit \( H(a) = H(a') \) schwer findbar
	\item Starke Kollisionsresistenz: Paar \( a \neq a' \) mit \( H(a) = H(a') \) schwer findbar
\end{itemize}

\paragraph{Integritätssicherung}
\begin{itemize}
	\item Daten sollen beim Empfänger genau so eintreffen, wie sie versendet wurden
	\item Manipulationen können nicht verhindert, nur erkannt werden -> Schwache Integrität
	\item \textbf{Message Authentication Code}:
	\begin{itemize}
    \item \emph{Ziel}: Empfänger erkennt Manipulation an empfangenen Daten
    \item \emph{Voraussetzung}: Alice und Bob haben gleichen symmetrischen Schlüssel
    \item \emph{Vorgehensweise}: Alice hängt Hash v. (Nachricht+Schlüssel) an Nachricht an
  \end{itemize}
	\item \textbf{Digitale Signatur}: Sichert Integrität
	\begin{itemize}
    \item \emph{Ziel}: Bob kann prüfen, dass wirklich Alice Dokument unterschrieben hat
    \item \emph{Vorgehensweise}: Hash des Dokuments mit privatem Schlüssel verschlüsseln, als Signatur mitsenden, entschlüsseln mit öffentlichem Schlüssel
  \end{itemize}
	\item \textbf{Digitales Zertifikat}: Sichert Authentizität
	\begin{itemize}
    \item Ziel: Verifizieren, dass jemand der ist, für den er sich ausgibt (öffentlicher Schlüssel tatsächlich zu ihm gehört)
    \item Problem: kann man nicht selbst überprüfen \( \leadsto \) verlassen auf Dritte
    \item Vorgehensweise: Überprüfung durch \emph{certificate authority} (CA), ID-Zertifikat = Authentifikation des öffentlichen Schlüssels
  \end{itemize}
\end{itemize}

\paragraph{E-Mail Sicherheit --- PGP (Pretty Good Privacy)}
\begin{itemize}
	\item SSL/TLS nur scheinbar sicher: Backend-Weiterverarbeitung unverschlüsselt
	\item E-Mail-Verschlüsselung PGP: Bietet Vertraulichkeit, Integrität, Authentizität
	\item Verwendet symmetrischen Schlüssel, der asymmetrisch Verschlüsselt zusammen mit Nachricht versendet wird
	\item \textbf{Problem}: Öffentliche Schlüssel müssen vor Versand bekannt sein
	\item Zugehörigkeit zur Mailadresse muss überprüfbar sein
	\item Bekanntgabe auf öffentlichen Schlüsselservern
	\item \textbf{Web of Trust}: Dezentraler, anarchischer Vertrauensansatz (ohne CAs)
	\begin{itemize}
    \item Transitive Überprüfung der Authentizität eines öffentlichen Schlüssels
    \item Nutzer bestimmt Vertrauen in Signaturen anderer Nutzer (Signatory Trust)
    \item Key Legitimacy bestimmt sich aus Anzahl der vertrauten Signaturen
  \end{itemize}
	\item In der Realität: Viele Probleme (Inseln)
\end{itemize}

\paragraph{Infrastruktursicherheit}
\begin{itemize}
	\item \textbf{Firewall}: Zugriffskontrolle durch Paketfilterung
	\begin{itemize}
    \item Teile des Netzes vor Eindringen unerwünschter Pakete schützen
    \item Netzbereiche (Intern, Internet, Demilitarized Zone) voneinander isolieren
    \item Zustandslos (IP, Port, Protokoll, Interface) mit Access Control List (ACL)
    \item Zustandsbehaftet (Überwacht TCP-Verbindungen)
    \item Application Layer Gateway: Filtern auf Basis von Nutzdaten (z.B. Username)
  \end{itemize}
	\item \textbf{Intrusion Detection and Prevention}:
	\begin{itemize}
    \item Bekannte Angriffe erkennen / verhindern
    \item Deep Packet Inspection (Analyse der Nutzdaten)
    \item Anomaly Detection (Alarm bei Abweichungen vom Normalverhalten)
  \end{itemize}
	\item \textbf{Organisatorische Maßnahmen}: Schulungen, Verantwortlicher, Notfallplan, Richtlinien, Datensicherung, \dots
\end{itemize}