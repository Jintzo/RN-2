\section{Einführung}

\paragraph{Was ist das Internet?}
\begin{itemize}
  \item \textbf{Komponentensicht}:
  \begin{enumerate}
    \item \emph{Hosts}: führen Netzwerkanwendungen aus
    \item \emph{Kommunikationsmedien}: Kupferkabel, Glasfaser, Funk
    \item \emph{Zwischensysteme}: Weiterleitung durch Router und Switches
  \end{enumerate}
  \item \textbf{Dienstsicht}: Infrastruktur, die Dienste für \emph{verteilte} Anwendungen bereitstellt
  \begin{itemize}
    \item \emph{Kommunikation} (Mail, Messaging, soziale Medien)
    \item \emph{Information} (Surfen)
    \item \emph{Unterhaltung} (Streaming, Spiele)
  \end{itemize}
  \item \textbf{Protokolle}: definieren Regel und Formate für Kommunikation
\end{itemize}

\paragraph{Rand des Internet}
\begin{itemize}
  \item \textbf{Endsysteme}: Clients, Server
  \item \textbf{Zugangsnetze}: Heimnetz, Mobiles Zugangsnetz, Unternehmensnetz
\end{itemize}

\paragraph{Kern des Internet}
\begin{itemize}
  \item \textbf{Pakete}: Voneinander unabhängige Einheiten für die Weiterleitung, werden durch das Netz zur Zielanwendung geleitet
  \item \textbf{Interne Struktur}:
  \begin{itemize}
    \item Zugangs-ISPs verbunden mit Globalen Tier 1 ISPs
    \item Verknüpft durch peering links und an IXPs (Internet Exchange Point)
    \item Dazu Content Provider Networks und Regionale Netze
  \end{itemize}
\end{itemize}

\paragraph{Internet-Historie}
\begin{itemize}
  \item \textbf{Paradigmenwechsel}: Telefonnetz (Leitungsvermittelt, Zentral)
  \begin{itemize}
    \item[\( \Rightarrow \)] Internet (Paketvermittelt, Dezentral)
  \end{itemize}
	\item \textbf{Anfang}: ARPAnet (1969), dann weitere Netze, Protokolle (zunächst Universitäten, dann zunehmende Kommerzialisierung)
	\item \textbf{Entwurfsprinzipien}: Minimalism/Autonomy, Best Effort Service, Stateless Routers, Decentralized Control
\end{itemize}