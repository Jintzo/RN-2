\section{Einführung}

\paragraph{Was ist das Internet?}
\begin{items}
  \item \textbf{Komponentensicht} \\*
    1. \emph{Hosts} -- führen Netzwerkanwendungen aus \\*
    2. \emph{Kommunikationsmedien} -- Kupferkabel, Glasfaser, Funk \\*
    3. \emph{Zwischensysteme} -- Weiterleitung durch Router und Switches
  \item \textbf{Dienstsicht} \\*
    \( \Rightarrow \) Infrastruktur, die Dienste für \emph{verteilte} Anwendungen bereitstellt \\*
    - \emph{Kommunikation} (Mail, Messaging, soziale Medien) \\*
    - \emph{Information} (Surfen) \\*
    - \emph{Unterhaltung} (Streaming, Spiele)
    \item \textbf{Protokolle} definieren Regel und Formate für Kommunikation
\end{items}

\paragraph{Rand des Internet}
\begin{items}
  \item \textbf{Endsysteme: } Clients, Server
  \item \textbf{Zugangsnetze: } Heimnetz, Mobiles Zugangsnetz, Unternehmensnetz
\end{items}

\paragraph{Kern des Internet}
\begin{items}
  \item \textbf{Pakete}: Voneinander unabhängige Einheiten für die Weiterleitung -- werden durch das Netz zur Zielanwendung geleitet\\*
  \item \textbf{Interne Struktur:}\\*
  - Zugangs-ISPs verbunden mit Globalen Tier 1 ISPs\\*
  - Verknüpft durch peering links und an IXPs (Internet Exchange Point)\\*
  - Dazu Content Provider Networks und Regionale Netze
\end{items}

\paragraph{Internet-Historie}
\begin{items}
	\item \textbf{Paradigmenwechsel}: Telefonnetz (Leitungsvermittelt, Zentral)\\* $\Rightarrow$ Internet (Paketvermittelt, Dezentral)
	
	\item \textbf{Anfang: ARPAnet} (1969), dann weitere Netze, Protokolle\\* 
		Zunächst Universitäten, dann zunehmende Kommerzialisierung
	\item \textbf{Entwurfsprinzipien:} Minimalism/Autonomy, Best Effort Service, Stateless Routers, Decentralized Control
\end{items}